% Options for packages loaded elsewhere
\PassOptionsToPackage{unicode}{hyperref}
\PassOptionsToPackage{hyphens}{url}
%
\documentclass[
]{book}
\usepackage{lmodern}
\usepackage{amssymb,amsmath}
\usepackage{ifxetex,ifluatex}
\ifnum 0\ifxetex 1\fi\ifluatex 1\fi=0 % if pdftex
  \usepackage[T1]{fontenc}
  \usepackage[utf8]{inputenc}
  \usepackage{textcomp} % provide euro and other symbols
\else % if luatex or xetex
  \usepackage{unicode-math}
  \defaultfontfeatures{Scale=MatchLowercase}
  \defaultfontfeatures[\rmfamily]{Ligatures=TeX,Scale=1}
\fi
% Use upquote if available, for straight quotes in verbatim environments
\IfFileExists{upquote.sty}{\usepackage{upquote}}{}
\IfFileExists{microtype.sty}{% use microtype if available
  \usepackage[]{microtype}
  \UseMicrotypeSet[protrusion]{basicmath} % disable protrusion for tt fonts
}{}
\makeatletter
\@ifundefined{KOMAClassName}{% if non-KOMA class
  \IfFileExists{parskip.sty}{%
    \usepackage{parskip}
  }{% else
    \setlength{\parindent}{0pt}
    \setlength{\parskip}{6pt plus 2pt minus 1pt}}
}{% if KOMA class
  \KOMAoptions{parskip=half}}
\makeatother
\usepackage{xcolor}
\IfFileExists{xurl.sty}{\usepackage{xurl}}{} % add URL line breaks if available
\IfFileExists{bookmark.sty}{\usepackage{bookmark}}{\usepackage{hyperref}}
\hypersetup{
  pdftitle={The Book of N3C},
  pdfauthor={The N3C Educational Committee},
  hidelinks,
  pdfcreator={LaTeX via pandoc}}
\urlstyle{same} % disable monospaced font for URLs
\usepackage{color}
\usepackage{fancyvrb}
\newcommand{\VerbBar}{|}
\newcommand{\VERB}{\Verb[commandchars=\\\{\}]}
\DefineVerbatimEnvironment{Highlighting}{Verbatim}{commandchars=\\\{\}}
% Add ',fontsize=\small' for more characters per line
\usepackage{framed}
\definecolor{shadecolor}{RGB}{248,248,248}
\newenvironment{Shaded}{\begin{snugshade}}{\end{snugshade}}
\newcommand{\AlertTok}[1]{\textcolor[rgb]{0.94,0.16,0.16}{#1}}
\newcommand{\AnnotationTok}[1]{\textcolor[rgb]{0.56,0.35,0.01}{\textbf{\textit{#1}}}}
\newcommand{\AttributeTok}[1]{\textcolor[rgb]{0.77,0.63,0.00}{#1}}
\newcommand{\BaseNTok}[1]{\textcolor[rgb]{0.00,0.00,0.81}{#1}}
\newcommand{\BuiltInTok}[1]{#1}
\newcommand{\CharTok}[1]{\textcolor[rgb]{0.31,0.60,0.02}{#1}}
\newcommand{\CommentTok}[1]{\textcolor[rgb]{0.56,0.35,0.01}{\textit{#1}}}
\newcommand{\CommentVarTok}[1]{\textcolor[rgb]{0.56,0.35,0.01}{\textbf{\textit{#1}}}}
\newcommand{\ConstantTok}[1]{\textcolor[rgb]{0.00,0.00,0.00}{#1}}
\newcommand{\ControlFlowTok}[1]{\textcolor[rgb]{0.13,0.29,0.53}{\textbf{#1}}}
\newcommand{\DataTypeTok}[1]{\textcolor[rgb]{0.13,0.29,0.53}{#1}}
\newcommand{\DecValTok}[1]{\textcolor[rgb]{0.00,0.00,0.81}{#1}}
\newcommand{\DocumentationTok}[1]{\textcolor[rgb]{0.56,0.35,0.01}{\textbf{\textit{#1}}}}
\newcommand{\ErrorTok}[1]{\textcolor[rgb]{0.64,0.00,0.00}{\textbf{#1}}}
\newcommand{\ExtensionTok}[1]{#1}
\newcommand{\FloatTok}[1]{\textcolor[rgb]{0.00,0.00,0.81}{#1}}
\newcommand{\FunctionTok}[1]{\textcolor[rgb]{0.00,0.00,0.00}{#1}}
\newcommand{\ImportTok}[1]{#1}
\newcommand{\InformationTok}[1]{\textcolor[rgb]{0.56,0.35,0.01}{\textbf{\textit{#1}}}}
\newcommand{\KeywordTok}[1]{\textcolor[rgb]{0.13,0.29,0.53}{\textbf{#1}}}
\newcommand{\NormalTok}[1]{#1}
\newcommand{\OperatorTok}[1]{\textcolor[rgb]{0.81,0.36,0.00}{\textbf{#1}}}
\newcommand{\OtherTok}[1]{\textcolor[rgb]{0.56,0.35,0.01}{#1}}
\newcommand{\PreprocessorTok}[1]{\textcolor[rgb]{0.56,0.35,0.01}{\textit{#1}}}
\newcommand{\RegionMarkerTok}[1]{#1}
\newcommand{\SpecialCharTok}[1]{\textcolor[rgb]{0.00,0.00,0.00}{#1}}
\newcommand{\SpecialStringTok}[1]{\textcolor[rgb]{0.31,0.60,0.02}{#1}}
\newcommand{\StringTok}[1]{\textcolor[rgb]{0.31,0.60,0.02}{#1}}
\newcommand{\VariableTok}[1]{\textcolor[rgb]{0.00,0.00,0.00}{#1}}
\newcommand{\VerbatimStringTok}[1]{\textcolor[rgb]{0.31,0.60,0.02}{#1}}
\newcommand{\WarningTok}[1]{\textcolor[rgb]{0.56,0.35,0.01}{\textbf{\textit{#1}}}}
\usepackage{longtable,booktabs}
% Correct order of tables after \paragraph or \subparagraph
\usepackage{etoolbox}
\makeatletter
\patchcmd\longtable{\par}{\if@noskipsec\mbox{}\fi\par}{}{}
\makeatother
% Allow footnotes in longtable head/foot
\IfFileExists{footnotehyper.sty}{\usepackage{footnotehyper}}{\usepackage{footnote}}
\makesavenoteenv{longtable}
\usepackage{graphicx,grffile}
\makeatletter
\def\maxwidth{\ifdim\Gin@nat@width>\linewidth\linewidth\else\Gin@nat@width\fi}
\def\maxheight{\ifdim\Gin@nat@height>\textheight\textheight\else\Gin@nat@height\fi}
\makeatother
% Scale images if necessary, so that they will not overflow the page
% margins by default, and it is still possible to overwrite the defaults
% using explicit options in \includegraphics[width, height, ...]{}
\setkeys{Gin}{width=\maxwidth,height=\maxheight,keepaspectratio}
% Set default figure placement to htbp
\makeatletter
\def\fps@figure{htbp}
\makeatother
\setlength{\emergencystretch}{3em} % prevent overfull lines
\providecommand{\tightlist}{%
  \setlength{\itemsep}{0pt}\setlength{\parskip}{0pt}}
\setcounter{secnumdepth}{5}
\usepackage{booktabs}
\usepackage[]{natbib}
\bibliographystyle{apalike}

\title{The Book of N3C}
\author{The N3C Educational Committee}
\date{2022-06-06}

\begin{document}
\maketitle

{
\setcounter{tocdepth}{1}
\tableofcontents
}
\hypertarget{preface}{%
\chapter{Preface}\label{preface}}

\emph{Something} needs to go in the index page\ldots{}

\hypertarget{example-chapter}{%
\chapter{Example Chapter}\label{example-chapter}}

This chapter is to exemplify various kinds of content for the book of N3C, and serve as a style guide.

\hypertarget{subheadings}{%
\section{Subheadings}\label{subheadings}}

Chapter titles should use a level-1 heading, topics and sub-topics should use level-2 and level-3 headings, but level-4 headings and beyond are discouraged.

Praesent id gravida erat. Aliquam volutpat leo vel orci blandit iaculis. Proin ornare ut libero eu euismod. Vivamus in tellus imperdiet, ullamcorper nibh in, facilisis ex. Proin tempus ligula sed enim hendrerit fringilla. Proin nec libero purus. Aliquam eget vestibulum ante, et volutpat lectus. Pellentesque placerat justo et dui porttitor posuere. Ut sit amet tellus in massa maximus tincidunt.

Cras lobortis tellus suscipit odio fringilla molestie. Praesent tellus ante, lacinia quis pulvinar eu, mollis eu mi. Donec quis tincidunt nunc. Nulla facilisi. Quisque sed nisi est. Aliquam erat volutpat. Phasellus venenatis tellus sit amet consectetur congue. Proin eu vestibulum ex. Fusce sit amet neque sapien.

\hypertarget{code-blocks}{%
\section{Code Blocks}\label{code-blocks}}

Code blocks may be formatted for Python, R, or SQL. Make ample use of code comments. Example python block:

\begin{Shaded}
\begin{Highlighting}[]
\ImportTok{import}\NormalTok{ pandas }\ImportTok{as}\NormalTok{ pd}

\CommentTok{# Both pneumonia_concept_set and condition_occurrence_sample are input as pandas dataframes in this example}
\KeywordTok{def}\NormalTok{ pneumonia_conditions2(pneumonia_concept_set, condition_occurrence):}

    \CommentTok{# Use an inner join to pull condition_occurrence entries matching concepts in the pneumonia_concept_set}
\NormalTok{    result }\OperatorTok{=}\NormalTok{ pd.merge(pneumonia_concept_set, condition_occurrence, how }\OperatorTok{=} \StringTok{"inner"}\NormalTok{, left_on }\OperatorTok{=} \StringTok{"concept_id"}\NormalTok{, right_on }\OperatorTok{=} \StringTok{"condition_concept_id"}\NormalTok{)}
    
    \CommentTok{# Calling print() results in output being logged to the "logs" tab}
    \BuiltInTok{print}\NormalTok{(result.describe())}

    \CommentTok{# Basic transforms should return a data frame (either pandas or spark)}
    \ControlFlowTok{return}\NormalTok{ result}
\end{Highlighting}
\end{Shaded}

Example R block:

\begin{Shaded}
\begin{Highlighting}[]
\CommentTok{# Both pneumonia_concept_set and condition_occurrence_sample are input as R dataframes in this example}
\NormalTok{pneumonia_conditions_result <-}\StringTok{ }\ControlFlowTok{function}\NormalTok{(pneumonia_concept_set, condition_occurrence) \{}
  
    \CommentTok{# Use an inner join to pull condition_occurrence entries matching concepts in the pneumonia_concept_set}
\NormalTok{    result <-}\StringTok{ }\NormalTok{pneumonia_concept_set }\OperatorTok\StringTok{ }
\StringTok{      }\KeywordTok{inner_join}\NormalTok{(condition_occurrence, }\DataTypeTok{by =} \KeywordTok{c}\NormalTok{(}\StringTok{"concept_id"}\NormalTok{ =}\StringTok{ "condition_concept_id"}\NormalTok{)) }\OperatorTok
\StringTok{      }\CommentTok{# we also convert the condition_source_value to an integer, since the }
\StringTok{      }\CommentTok{# conversion from spark's long type results in an R integer64 type, incompatible with}
\StringTok{      }\CommentTok{# many functions}
\StringTok{      }\KeywordTok{mutate}\NormalTok{(}\DataTypeTok{condition_source_value =} \KeywordTok{as.integer}\NormalTok{(condition_source_value))}
  
\NormalTok{    p <-}\StringTok{ }\KeywordTok{ggplot}\NormalTok{(result) }\OperatorTok{+}\StringTok{ }
\StringTok{      }\KeywordTok{geom_histogram}\NormalTok{(}\KeywordTok{aes}\NormalTok{(}\DataTypeTok{x =}\NormalTok{ condition_end_date }\OperatorTok{-}\StringTok{ }\NormalTok{condition_start_date))}
    
    \CommentTok{# an explicit call to plot() is needed to display the resulting ggplot}
    \KeywordTok{plot}\NormalTok{(p)}
    
    \CommentTok{# the str() function is useful for identifying data contents and types}
    \KeywordTok{print}\NormalTok{(}\KeywordTok{str}\NormalTok{(result))}
    
    \KeywordTok{return}\NormalTok{(result)}
\NormalTok{\}}
\end{Highlighting}
\end{Shaded}

And an example SQL block:

\begin{Shaded}
\begin{Highlighting}[]
\CommentTok{-- filter the concept_set_members table input to select a single concept set version}
\KeywordTok{SELECT} \OperatorTok{*}
\KeywordTok{FROM}\NormalTok{ concept_set_members}
\KeywordTok{WHERE}\NormalTok{ codeset_id }\OperatorTok{=} \DecValTok{761347708}
\end{Highlighting}
\end{Shaded}

Output blocks (demonstrating logged output) should be uncolored:

\begin{verbatim}
'data.frame':   19502 obs. of  27 variables:
 $ codeset_id                   : int  761347708 761347708 761347708 761347708 761347708 761347708 761347708 761347708 761347708 761347708 ...
 $ concept_id                   : int  255848 255848 255848 255848 255848 255848 255848 255848 255848 255848 ...
 $ concept_set_name             : chr  "Non-influenza pneumonia [training]" "Non-influenza pneumonia [training]" "Non-influenza pneumonia [training]" "Non-influenza pneumonia [training]" ...
 $ is_most_recent_version       : logi  TRUE TRUE TRUE TRUE TRUE TRUE ...
 $ version                      : int  1 1 1 1 1 1 1 1 1 1 ...
 $ concept_name                 : chr  "Pneumonia" "Pneumonia" "Pneumonia" "Pneumonia" ...
 $ archived                     : logi  FALSE FALSE FALSE FALSE FALSE FALSE ...
 $ person_id                    : int  148 833 1088 1238 1829 2142 2659 3749 8086 9465 ...
...
\end{verbatim}

\hypertarget{bullets-and-lists}{%
\section{Bullets and Lists}\label{bullets-and-lists}}

Unordered and ordered lists may be used, but they should be used sparingly and not a replacement for reader-friendly prose.

\begin{itemize}
\tightlist
\item
  Item 1

  \begin{itemize}
  \tightlist
  \item
    Subitem 1.1
  \item
    Subitem 1.2
  \end{itemize}
\item
  Item 2
\item
  Item 3
\end{itemize}

Ordered bullet list:

\begin{enumerate}
\def\labelenumi{\arabic{enumi}.}
\tightlist
\item
  Item 1

  \begin{enumerate}
  \def\labelenumii{\arabic{enumii}.}
  \tightlist
  \item
    Subitem 1.1
  \item
    Subitem 1.2
  \end{enumerate}
\item
  Item 2
\item
  Item 3
\end{enumerate}

\hypertarget{equations}{%
\section{Equations}\label{equations}}

We don't anticipate much need for equations, but just in case here's an example:

\begin{equation}
  f\left(k\right) = \binom{n}{k} p^k\left(1-p\right)^{n-k}
  \label{eq:binom}
\end{equation}

\hypertarget{callouts}{%
\section{Callouts}\label{callouts}}

Callout blocks may be used to emphasize certain points, provide additional information, or other uses.

This is a Note callout.

\hypertarget{markdown-in-callouts}{%
\subsection{Markdown in callouts}\label{markdown-in-callouts}}

Markdown formatting can be used in callouts. Here's an example code block:

\begin{Shaded}
\begin{Highlighting}[]
\CommentTok{-- filter the concept_set_members table input to select a single concept set version}
\KeywordTok{SELECT} \OperatorTok{*}
\KeywordTok{FROM}\NormalTok{ concept_set_members}
\KeywordTok{WHERE}\NormalTok{ codeset_id }\OperatorTok{=} \DecValTok{761347708}
\end{Highlighting}
\end{Shaded}

Other callout types include \texttt{.rmdcaution}:

This is a Caution callout.

\texttt{.rmdimportant}:

This is an Important callout.

\texttt{.rmdtip}:

This is a Tip callout.

and \texttt{.rmdwarning}:

This is a Warning callout.

\hypertarget{screenshots-and-images}{%
\section{Screenshots and Images}\label{screenshots-and-images}}

Here's a screenshot:

\hypertarget{title}{%
\chapter{Title}\label{title}}

Content

\hypertarget{title-1}{%
\chapter{Title}\label{title-1}}

Content

\hypertarget{title-2}{%
\chapter{Title}\label{title-2}}

Content

\hypertarget{title-3}{%
\chapter{Title}\label{title-3}}

Content

\hypertarget{title-4}{%
\chapter{Title}\label{title-4}}

Content

\hypertarget{title-5}{%
\chapter{Title}\label{title-5}}

Content

\hypertarget{title-6}{%
\chapter{Title}\label{title-6}}

Content

\hypertarget{title-7}{%
\chapter{Title}\label{title-7}}

Content

\hypertarget{title-8}{%
\chapter{Title}\label{title-8}}

Content

\hypertarget{title-9}{%
\chapter{Title}\label{title-9}}

Content

\hypertarget{title-10}{%
\chapter{Title}\label{title-10}}

Content

\hypertarget{title-11}{%
\chapter{Title}\label{title-11}}

Content

\hypertarget{title-12}{%
\chapter{Title}\label{title-12}}

Content

  \bibliography{book.bib,packages.bib}

\end{document}
